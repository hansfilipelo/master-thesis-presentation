%% Copyright 2015 by Clas Veibäck, Automatic Control, Linköping University

%% Notes on usage
%% With latex and sansserif fonts, pdfLaTeX can be used
%% With standard font, LuaLaTeX needs to be used and Calibri and Georgia need to be installed
%% With LiU font, LuaLaTeX needs to be used and Korolev LiU and Georgi need to be installed
%% With calibri font, LuaLaTeX needs to be used and Calibri need to be installed

\documentclass[noamsthm, english]{beamer}%

\usepackage[T1]{fontenc}%
\usepackage{babel}%
\usepackage{blindtext}%

%%
%% Commands
%%
\newcommand{\function}[3]{%
	\begin{center}%
	{\usebeamerfont{header}{{\usebeamercolor[fg]{orange inverse}\textbackslash #1}}%
	{\usebeamercolor[fg]{green inverse}{[#2]}}%
	\{#3\}}%
	\end{center}%
}

\newcommand{\shortfunction}[2]{%
	\begin{center}%
	{\usebeamerfont{header}{{\usebeamercolor[fg]{orange inverse}\textbackslash #1}}%
	\{#2\}}%
	\end{center}%
}

\newcommand{\shortinlinefunction}[2]{%
	{\usebeamerfont{header}{{\usebeamercolor[fg]{orange inverse}\textbackslash #1}}%
	\{#2\}}%
}

\newcommand{\macro}[1]{{\usebeamerfont{header}{{\usebeamercolor[fg]{orange inverse}\textbackslash #1}}}
}

\newcommand{\environ}[3]{%
	\begin{center}%
	{\usebeamerfont{header}%
	{{\usebeamercolor[fg]{orange inverse}\textbackslash begin \{#1\}}}%
	{\usebeamercolor[fg]{green inverse}[#2]}%
	#3%
	{{\usebeamercolor[fg]{orange inverse}\textbackslash end \{#1\}}}}%
	\end{center}%
}

\newcommand{\code}[2][Example Code]{\begin{block}{#1}\small \vspace{-1em}\begin{semiverbatim}#2\end{semiverbatim}\vspace{-1em}\end{block}}%

\newenvironment{codeblock}[1][Example Code]
    {\begin{block}{#1}\footnotesize\vspace{-1em}\begin{semiverbatim}}%
    {\end{semiverbatim}\vspace{-1em}\end{block}}%

%%
%%  Theme
%%
\usetheme[%
%% Standard colors {blue, green, turquoise} for title, outline and end pages
titlecolor=blue,%
outlinecolor=green,%
endcolor=turquoise,%
%% Complementary colors {orange, purple, yellow, gray} for some details
complementary=orange,%
%% All colors {blue, green, turquoise, orange, purple, yellow, gray} for blocks
blockcolor=blue,%
%% Font schemes {latex, standard, sansserif, calibri, liu} to use
font=liu,%
%% Select color themes {highcontrast, superhighcontrast, dark} with high contrast or add handout option in documentclass
%highcontrast,%
%superhighcontrast,%
%% Remove outline before each section and end page with {nooutline, noendpage}
%nooutline, %
%noendpage,%
%% Add navigation symbols with {navigation}
%navigation,%
%% Add total number of frames to frame count with {totalframes}
totalframes,%
% Remove parts of header with {noheadertitle, noheaderauthor, noheaderdate, noheadernumber, minimalheader, noheader}
%noheadertitle,%
%noheaderauthor,%
%noheaderdate,%
%noheadernumber,%
%% Show outline in two columns
outlinecolumns=2,%
]{LiU}%

%% Add a text on end page, leave blank to add author
\finaltext{Clas Veibäck \\[-0.5em]%
	 \small \href{clas.veiback@liu.se}%
	 {clas.veiback@liu.se}\vspace{0.5em}
}

%%
%% Presentation information
%%
\title[LiU Slides Documentation]{LiU Slides Theme Documentation}%
\subtitle{A guide to getting started with LiU slides}%
\author{Clas Veibäck}%
\institute{Automatic Control\\%
Department of Electrical Engineering\\%
Linköping University}%
\date{\today}%

%%
%% Presentation
%%
\begin{document}%

\maketitle
\makeoutline

%%
\section{Introduction}
%%

\begin{frame}
\frametitle{Disclaimer}
This theme is approved by the Communications and Marketing Division at Linköping University, but is provided ``as is'', without warranty of any kind by the author or the Communications and Marketing Division.
\\~\\
Bugs, comments and suggestions for improvements can be sent to Clas Veibäck at \href{mailto:clas.veiback@liu.se}{clas.veiback@liu.se}.
\end{frame}

\begin{frame}
\frametitle{Introduction}
This theme is based on the PowerPoint template provided by the Communications and Marketing Division and adheres to the graphic profile at Linköping University.
\\~\\
The theme is provided to simplify the process of making a LiU presentation.
\end{frame}

%%
\section{Installation}
%%

\begin{frame}
\frametitle{Installation}
The files in the theme should be installed into the \emph{texmf} tree of your \emph{TeX} installation. The installation also includes a template that can be added to your \emph{TeX} installation.
\\~\\
See the documentation for details.
\end{frame}

%%
\section{Features}
%%
\subsection{Using Theme}
%%

\begin{frame}[fragile]
\frametitle{Using Theme}
To use the LiU theme, the command \function{usetheme}{<options>}{LiU} is added to the pre-amble of your beamer presentation.
\code[\textbackslash usetheme example]{\\usetheme[font=standard]\{LiU\}}
\end{frame}

%%
\subsection{Colors}
%%

\begin{frame}
\frametitle{Colors}
There are three standard colors with tones
{
\begin{itemize}
\item \colorbox{LiUblue}{\color{white}blue}, \colorbox{LiUblue3}{\color{black}lightblue},  \colorbox{LiUblue5}{\color{black}brightblue} and \colorbox{LiUblue0}{\color{white}darkblue}
\item \colorbox{LiUgreen}{\color{white}green}, \colorbox{LiUgreen3}{\color{black}lightgreen},  \colorbox{LiUgreen5}{\color{black}brightgreen} and \colorbox{LiUgreen0}{\color{white}darkgreen}
\item \colorbox{LiUturquoise}{\color{white}turquoise}, \colorbox{LiUturquoise3}{\color{black}lightturquoise},  \colorbox{LiUturquoise5}{\color{black}brightturquoise} and \colorbox{LiUturquoise0}{\color{white}darkturquoise}
\end{itemize}
}
and four complementary colors with tones
{
\begin{itemize}
\item {\color{LiUorange} orange}, {\color{LiUorange3} lightorange}, {\color{LiUorange5} brightorange}
\item {\color{LiUpurple} purple}, {\color{LiUpurple3} lightpurple}, {\color{LiUpurple5} brightpurple}
\item {\color{LiUyellow} yellow},{\color{LiUyellow3} lightyellow}, {\color{LiUyellow5} brightyellow}
\item {\color{LiUgray} gray}, {\color{LiUgray3} lightgray}, {\color{LiUgray5} brightgray}
\end{itemize}
}
\end{frame}

\begin{frame}
\frametitle{Color Options}
Colors can be accessed using the command \function{usebeamercolor}{bg}{<color>}
\code[\textbackslash usebeamercolor example]{\{\\usebeamercolor[bg]\{gray\} This text is gray.\}}
\end{frame}

%%
\subsection{LiU Frame}
%%

\begin{frame}[fragile]
\frametitle{LiU Frame}
A solid color frame can be inserted using the environment
%\environ{LiUframe}{<color>}{<content>}
\function{LiUframe}{<color>}{<content>}
Any of the standard colors and their tones can be used.\\~\\
The macro \macro{LiUframewidth} gives the width of the page and the macro \macro{LiUframemargin} gives the margin of a LiU frame.
\\~\\
Do not use a frame title for these frames.
\end{frame}

\begin{frame}[fragile]
\frametitle{LiU Frame}
\begin{codeblock}[LiUframe example]
\\LiUframe[blue]\{
\hspace{1em}\\raggedright
\hspace{1em}\\usebeamerfont\{title\}
\hspace{2em}``You're the all~singing, all~dancing
\hspace{2em}crap of the world.''\\par
\hspace{1em}\{\\raggedleft \\usebeamerfont\{subtitle\}
\hspace{2em}--- Tyler Durden\\par\}
\hspace{1em}\\vspace\{20mm\}
\}
\end{codeblock}
\end{frame}

\LiUframe[blue]{
	\raggedright
	\usebeamerfont{title}``You're the all~singing, all~dancing crap of the world.''\par%
	{\raggedleft \usebeamerfont{subtitle}--- Tyler Durden\par}%

	\vspace{20mm}
}

%%
\subsection{Special Frames}
%%

\begin{frame}[fragile]
\frametitle{Title Page}
A title page can be inserted using \macro{maketitle} outside a frame or \macro{titlepage} within a LiU frame.
\\~\\
The title, subtitle, author and institute are given by the standard pre-amble commands.
\end{frame}

%% Undo reset of framenumber
\newcounter{tempframenumber}
\setcounter{tempframenumber}{\value{framenumber}}
\maketitle
\setcounter{framenumber}{\value{tempframenumber}}
\addtocounter{framenumber}{1}

\begin{frame}[fragile]
\frametitle{Outline Page}
An outline page can be inserted using \macro{makeoutline} outside a frame or \macro{outlinepage} within a LiU frame.
\\~\\
The optional argument to \macro{makeoutline} is passed on to \macro{tableofcontents}\hspace{-0.5ex}.
\\~\\
An outline of the next section is shown at the beginning of each section.
\end{frame}

\makeoutline

\begin{frame}[fragile]
\frametitle{End Page}
An end page can be inserted using \macro{makeend} outside a frame or \macro{endpage} within a LiU frame.
\\~\\
Normally the author is shown, but to modify the text on the end page the command \shortfunction{finaltext}{message} can be added in the pre-amble after the command \shortinlinefunction{usetheme}{LiU}.
\\~\\
An end page is automatically shown as the final slide and should normally not be added by the author.
\end{frame}
\finaltext{Clas Veibäck\\[-0.5em] {\small \href{mailto:clas.veiback@liu.se}{clas.veiback@liu.se}}\vspace{0.5em}}
\begin{frame}[fragile]
\frametitle{End Page}
\begin{codeblock}[\textbackslash finaltext example]
\\finaltext\{%
\hspace{1em}Clas Veibäck\\\\%
\hspace{1em}\\small \\href\{mailto:clas.veiback@liu.se\}%
\hspace{1em}\{clas.veiback@liu.se\}%
\}
\end{codeblock}
\end{frame}

\makeend

%%
\section{Theme Options}
%%
\subsection{Color Options}
%%

\begin{frame}
\frametitle{Color Options}
\begin{block}{Color options}
\centering \footnotesize
\begin{description}[complementary]
\item[titlecolor] Color of title page, default is blue
\item[outlinecolor] Color of outline page, default is blue
\item[endcolor] Color of outline page, default is blue
\item[complementary] Color of some details, e.g. alerted text, default is orange
\item[blockcolor] Color of a block, such as this one, default is blue
\end{description}
\end{block}
\end{frame}

\begin{frame}[fragile]
\frametitle{Color Options}
\begin{codeblock}[Color options example]
\\usetheme[%
\hspace{1em}titlecolor=blue,%
\hspace{1em}outlinecolor=green,%
\hspace{1em}endcolor=turquoise,%
\hspace{1em}complementary=purple,%
\hspace{1em}blockcolor=green%
]\{LiU\}
\end{codeblock}
\end{frame}

%%
\subsection{Contrast Options}
%%

\begin{frame}
\frametitle{Contrast Options}
\begin{block}{Contrast options}
\centering \footnotesize
\begin{description}[superhighcontrast]
\item[highcontrast] Use the light version of the color theme and black foreground
\item[superhighcontrast] Use the bright version of the color theme and black foreground
\item[dark] Use the dark version of the color theme and white foreground
\item[handout] Remove theme colors, suitable for printing
\end{description}
\end{block}
 \alert{Note: {\footnotesize handout} is an option of \macro{documentclass}}
\end{frame}

\begin{frame}[fragile]
\frametitle{Contrast Options}
\begin{codeblock}[Contrast options example]
\\documentclass[english,handout]\{beamer\}

\\usetheme[%
\hspace{1em}titlecolor=blue,%
\hspace{1em}outlinecolor=green,%
\hspace{1em}endcolor=turquoise,%
\hspace{1em}highcontrast%
]\{LiU\}
\end{codeblock}
\end{frame}

%%
\subsection{Font Options}
%%

\begin{frame}
\frametitle{Font Options}
\begin{block}{Font options}
\centering \footnotesize
\begin{description}[standard]
\item[latex] Use native latex fonts, pdfLaTeX can be used, default
\item[standard] Use Calibri and Georgia
\item[sansserif] Use native latex with normal text in sans serif
\item[calibri] Use Calibri for all text
\item[liu] Use Korolev LiU and Georgia
\end{description}
\end{block}
\alert{Note: All non-latex fonts are required to be installed and can only be used with LuaLaTeX.}
\end{frame}

\begin{frame}[fragile]
\frametitle{Font Options}
\begin{codeblock}[Font options example]
\\usetheme[font=standard]\{LiU\}
\end{codeblock}
\end{frame}

%%
\subsection{Header Options}
%%

\begin{frame}
\frametitle{Header Options}
\begin{block}{Header options}
\centering \footnotesize
\begin{description}[noheadernumber]
\item[noheadertitle] Remove title in header
\item[noheaderauthor] Remove author in header
\item[noheaderdate] Remove date in header
\item[noheadernumber] Remove frame number in header
\item[totalframes] Add total number of frames in header
\end{description}
\end{block}
\end{frame}

\begin{frame}[fragile]
\frametitle{Header Options}
\begin{codeblock}[Header options example]
\\usetheme[%
\hspace{1em}noheaderdate,%
\hspace{1em}totalframes%
]\{LiU\}
\end{codeblock}
\end{frame}

%%
\subsection{Footer Options}
%%

\begin{frame}
\frametitle{Footer Options}
\begin{block}{Footer options}
\centering \footnotesize
\begin{description}[navigation]
\item[navigation] Add navigation symbols to the footer
\end{description}
\end{block}
\end{frame}

\begin{frame}[fragile]
\frametitle{Footer Options}
\begin{codeblock}[Footer options example]
\\usetheme[navigation]\{LiU\}
\end{codeblock}
\end{frame}

%%
\subsection{Special Page Options}
%%

\begin{frame}
\frametitle{Special Page Options}
The following are options for modifying the appearance of the outline page and end page.
\begin{block}{Special page options}
\centering \footnotesize
\begin{description}[noheadernumber]
\item[noendpage] Remove end page
\item[nooutline] Do not insert outline at start of each section
\item[outlinecolumns={\it n}] Use {\it n} columns for the outline page
\end{description}
\end{block}
\end{frame}

\begin{frame}[fragile]
\frametitle{Special Page Options}
The following is an example of using the special page options.
\begin{codeblock}[Special page options]
\\usetheme[%
\hspace{1em}showinstitute,%
\hspace{1em}noendpage,%
\hspace{1em}outlinecolumns=2%
]\{LiU\}
\end{codeblock}
\end{frame}

\end{document}
